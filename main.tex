\documentclass[lang=cn,10pt,twoside]{elegantbook}

\title{微积分入门}
\subtitle{Int256 带你一起学习微积分}

\author{Int256}
% \institute{组织}
\date{\today}
\version{1.5.1b}
% \bioinfo{打印提示}{若要打印,请选择双面打印}

\extrainfo{给逝去的岁月以眼泪,但不要忘记给未来的时光以期待!}

\setcounter{tocdepth}{3}

% \logo{logo-blue.png}
\cover{math.jpg}

% 本文档命令
\usepackage{array}

% \newcommand{\ccr}[1]{\makecell{{\color{#1}\rule{1cm}{1cm}}}}
% \setCJKfamilyfont{NLXJT}{NLXJT}
% \newcommand{\hmz}{\CJKfamily{NLXJT}}
% \definecolor{customcolor}{RGB}{32,178,170}
% \colorlet{coverlinecolor}{customcolor}

\begin{document}

\maketitle
\frontmatter

\tableofcontents

\mainmatter

\chapter{序}

写这本书其实纯属是突发奇想。主要原因是很多同学认为微积分很难。若想打印本书,强烈建议双面打印。

如果你发现任何问题,笔者的联系方式(邮箱)如下,欢迎反馈。也可以加入 QQ 群提问。

\begin{itemize}
  \item 邮箱:\email{alanlym@yeah.net}
  \item 洛谷:\href{https://www.luogu.com.cn/user/344700}{int256}
  \item QQ:179401445,TripleIntegrals。读者也可以管笔者叫三重积分
  \item B 站:\href{https://space.bilibili.com/565436033}{TripleIntegrals}
  \item QQ 群:\href{https://jq.qq.com/?_wv=1027&k=lpfM484S}{962035496}
\end{itemize}

本书的章后习题将在最后统一讲解。
由于前面的几章内容比较简单,所以习题会从《高等数学》等的经典书籍中摘录一些。

\section{简介}

本书可能有许多不严谨的地方,基本都是按照笔者的理解写的。正规定义会予以特殊标注。不过一般考试不会考怎么定义的。而且一些书上的定义着实看不懂。

\subsection{符号约定}
\begin{itemize}
  \item $\forall x$ 代表对于任意的 $x$。
  
  符号 $\forall x: p(x)$(如 $\forall x>0$),代表对于任意的符合 $p(x)$ 的 $x$(如对于任意大于 $0$ 的 $x$)。
  \item $\exists x$ 代表存在 $x$。
  
  符号 $\exists x: p(x)$(如 $\exists x > 0$),代表存在符合 $p(x)$ 的 $x$(如大于零的 $x$)。
  \item 符号
  \begin{equation*}
    \max\{a_1, a_2, \dots, a_n\}
  \end{equation*}
  表示 $a_1, a_2, \dots, a_n$ 中最大的一个。
  \item 符号
  \begin{equation*}
    \min\{a_1, a_2, \dots, a_n\}
  \end{equation*}
  表示 $a_1, a_2, \dots, a_n$ 中最小的一个。
  \item 符号 $[x]$ 表示 $x$ 的整数部分,即不超过 $x$ 的最大整数。
  \item 符号 $\square$ 表示证明完毕,或显然(证明过程极其简单),读者可以自行尝试证明。
  \item $\sum$ 为连加符号、$\prod$ 为连乘符号。用法及定义十分简单,读者可以自行查询。
  \item 符号 $\subset$ 表示包含(不限于真包含)。即如果集合 $A$ 的任何元素都是集合 $B$ 的元素,则称 $A \subset B$。

  \item 用空心字母 $\mathbb{N, Z, Q, R, C}$ 来表示全体自然数(含 $0$)、整数、有理数、实数、复数的集合。

  \item 一般的,一个集合右上角加上 $^*$ 表示这个集合除了 $0$ 以外的其他元素构成的集合。类似的,一个集合的右上角加上 $^+$ 表示这个集合的大于 $0$ 的元素构成的集合。

  \item 通俗地说,$a^+$ 或 $a+$ 表示比 $a$ 大一个极小的数,这个数无限接近于 $0$,取值可以看作 $a$ 但是大于 $a$。类似的,$a^-$ 或 $a-$ 表示比 $a$ 小一个极小的数,这个数无限接近于 $0$,取值可以看作 $a$ 但是小于 $a$。具体的定义后面会讲到。

  \item 其余符号将在第一次出现时给予说明。
\end{itemize}

\chapter{预备知识}
\begin{introduction}
  \item 差集运算
  \item 关于“空集”
  \item 存在与任意
  \item 映射
  \item 函数
  \item 逆映射
  \item 自然定义域
  \item 复合函数
  \item 函数的有界性
  \item 函数的奇偶性
  \item 函数的单调性
  \item 函数的周期性
\end{introduction}
本章讲解一些数学分析相关的基础知识。

\section{集合与逻辑符号}

\subsection{集合}
对于两个集合的 $\setminus$ 运算,表示求它们的差集,例如 $A \setminus B$ 表示属于集合 $A$ 但不属于集合 $B$ 的元素组成的集合。其余的高中必修一的课本有讲解过。这里只讲解一个常见误区:$\varnothing$ 是一个集合,而不是一个“一般性”的元素。

\subsection{逻辑符号}
设 $\alpha$ 和 $\beta$ 是两个判断。如果当 $\alpha$ 成立时 $\beta$ 也成立,我们就说 $\alpha$ 能够推出 $\beta$,或者 $\alpha$ \emph{蕴含} $\beta$。记为 $\alpha \Rightarrow \beta$。例如 $x \in \mathbb{R} \Rightarrow x \geq 0$。

如果 $\beta\Rightarrow\alpha$ 且 $\alpha\Rightarrow\beta$,我们就说 $\alpha$ 和 $\beta$ \emph{等价},或者说 $\alpha$ 和 $\beta$ 互为\emph{充要条件}。记为 $\alpha \Leftrightarrow \beta$。例如对于 $x \in \mathbb{R}$, $x > 0 \Leftrightarrow \dfrac 1x > 0$。

设 $\alpha(x)$ 是涉及 $x \in E$ 的一个判断,我们用记号

\begin{equation*}
  (\exists x \in E)(\alpha(x))
\end{equation*}

表示“存在 $x \in E$ 使得 $\alpha(x)$ 成立”。例如:

\begin{equation*}
  (\exists n \in \mathbb{N})(n^2 -4n + 4 > 0).
\end{equation*}

设 $\beta(x)$ 是涉及 $x \in E$ 的一个判断,我们用记号

\begin{equation*}
  (\forall x \in E)(\beta(x))
\end{equation*}

或者 $\beta(x), \forall x \in E$ 表示“对于一切 $x \in E$,$\beta(x)$ 成立”。例如:

\begin{equation*}
  (\forall n \in \mathbb{R})(n^2 \geq 0)
\end{equation*}

或者 $n^2 \geq 0, \forall n \in \mathbb{R}$。

\section{函数与映射}

\subsection{基础定义}

其实这个没啥好讲的。不过考虑到数学分析和高等数学的教材中都有提到,所以再讲一遍具体的定义。

我们知道\emph{函数}的定义中提到了“\emph{一一对应}”。即自变量 $x$ 所取的任意一个确定的值,决定了因变量 $y$ 的唯一确定的值与之对应。采用集合论的术语对这说法进一步地概括并描述,就得到了映射的概念。

设 $A$ 和 $B$ 都是集合,我们把 $A$ 的元素与 $B$ 的元素之间的对应关系 $f$ 叫做一个\emph{映射}。如果按照这种对应关系,对集合 $A$ 中的任何一个元素 $a$,有集合 $B$ 中的唯一一个元素 $b$ 与之对应。$f$ 是从 $A$ 到 $B$ 的一个映射,一般记为 $f:A\rightarrow B$。
按照对应关系 $f$,由 $A$ 中的元素 $a$ 所决定的 $B$ 中的唯一元素 $b$ 记为 $f(a)$。有时候,我们用记号 $a \mapsto b$ 表示元素之间的对应。例如,设 $A=\mathbb R, B = \mathbb R$,而映射 $f: A \rightarrow B$ 定义为 $f(x) = x^2$, 则这映射规定了元素之间这样的对应关系:$f: x \mapsto x^2$。


设 $f:A\rightarrow B$ 是一个映射,$C \subset A$, $D \subset B$, 我们把集合 $f(C)=\{f(x) \mid x \in C\}(\subset B)$ 叫做集合 $C$ 经过映射 $f$ 的\emph{像集},并把集合 $f^{-1}(D)=\{x \mid f(x) \in D\}(\subset A)$ 叫做集合 $D$ 关于映射 $f$ 的\emph{原像集}。

如果 $A \subset \mathbb R, B = \mathbb R$, 那么从 $A$ 到 $B$ 的映射就是通常的一元函数。但映射的概念实际上更广,以后将会遇到更广泛的映射的例子。不过现在我们主要关心的是函数。

也就是说,一个函数的所有的自变量取值组成了它的定义域,即原像,而它的值组成了它的像集。如果对于 $f: A \rightarrow B$,对于任意的两个 $x, y \in A$, $f(x) \neq f(y)$,则称 $f$ 为 $A$ 到 $B$ 的\emph{单射}。

一个函数的逆映射就是从 $B$ 的每个值返回到 $A$ 的一个法则。即如果对于 $f: A \rightarrow B$,$f^{-1}: B \rightarrow A$,对每个 $y \in B$, $f^{-1}(y) = x$,$f(x) = y$。则这个映射 $f^{-1}$ 称为 $f$ 的\emph{逆映射}。

使得一个函数的值有意义的所有可行自变量的取值组成的集合是这个函数的\emph{自然定义域}。

设有两个映射 $g: A \rightarrow B_1, f: B_2 \rightarrow C$, 其中 $B_1 \subset B_2$, 则由映射 $g$ 和 $f$ 可以定出一个从 $A$ 到 $C$ 的对应法则,它将每个 $a \in A$ 映成 $f[g(a)] \in C$。这个对应法则确定了一个从 $A$ 到 $C$ 的映射,这个映射就称为映射 $g$ 和 $f$ 构成的\emph{复合映射},记为 $f\circ g$,即

\begin{equation*}
  f \circ g: A \rightarrow C, (f \circ g)(x) = f[g(x)], x \in A
\end{equation*}

显然由复合映射的定义可以得出,$g$ 的值域必须包含在 $f$ 的定义域内。

\subsection{函数的有界性}

设函数 $f(x)$ 的定义域为 $D$, 数集 $X \subset D$。如果存在 $N_1$ 使得 $f(x) \leq N_1$ 对于任一 $x \in X$ 都成立,那么称函数 $f(x)$ 在 $X$ 上有\emph{上界},而 $N_1$ 称为函数 $f(x)$ 在 $X$ 上的一个上界。

如果存在数 $N_2$ 使得 $f(x) \geq N_2$ 对于任一 $x \in X$ 都成立,那么称函数 $f(x)$ 在 $X$ 上有\emph{下界},而 $N_2$ 称为函数 $f(x)$ 在 $X$ 上的一个下界。

函数的\emph{最小上界}定义为最小的一个 $N_1$,\emph{最大下界}定义为最大的一个 $N_2$。

如果存在整数 $K$ 使得 $|f(x)| \leq K, \forall x \in X$,那么称函数 $f(x)$ 在 $X$ 上有\emph{上界}。如果这样的 $K$ 不存在,就称函数 $f(x)$ 在 $X$ 上\emph{无界}。

\subsection{函数的奇偶性}

如果 $f(x)$ 满足 $f(x) = -f(-x), \forall x$,则称函数 $f(x)$ 为\emph{奇函数}。

如果 $f(x)$ 满足 $f(x) = f(-x), \forall x$, 则称函数 $f(x)$ 为\emph{偶函数}。

显然,奇偶函数的定义域都是“对称”的。

如果上面两个条件都不满足,那么称函数 $f(x)$ 是\emph{非奇非偶}函数。

\subsection{函数的单调性}

设 $f: D \rightarrow T$, 区间 $I \subset D$。如果对于区间 $I$ 上的任意两点 $x_1, x_2$,当 $x_1 < x_2$ 时,恒有 $f(x_1) < f(x_2)$,那么称函数 $f$ 在区间 $I$ 上是\emph{单调增加}的。
如果对于区间 $I$ 上的任意两点 $x_1, x_2$,当 $x_1 < x_2$ 时,恒有 $f(x_1) > f(x_2)$,那么称函数 $f$ 在区间 $I$ 上是\emph{单调减少}的。

单调增加和单调减少的函数统称为\emph{单调函数}。

\emph{增函数}(\emph{单调递增})指在函数的定义域上函数是单调增加的。\emph{减函数}(\emph{单调递减})指在函数的定义域上函数是单调减少的。

\subsection{函数的周期性}

设函数 $f(x)$ 的定义域为 $D$。如果存在一个正数 $l$,使得对于任一 $x \in D$ 都有 $(x \pm l) \in D$, 且 $f(x+l)=f(x)$ 恒成立,那么称 $f(x)$ 为\emph{周期函数},$l$ 称为 $f(x)$ 的\emph{周期},通常我们说周期函数的周期是指函数的\emph{最小正周期}。

\newpage

\begin{problemset}
  \item 函数 $f(x) = \lg x^2$ 和函数 $g(x) = 2 \lg x$ 是否相同?为什么?
  \item 求出并证明函数 $f(x) = \frac{x}{1-x}$ 在 $(-\infty, 1)$ 上的单调性。
  \item 求出并证明函数 $f(x) = x + \ln x$ 在 $(0, +\infty)$ 上的单调性。
  \item 设 $f(x)$ 为定义在 $(-l, l)$ 上的奇函数,若 $f(x)$ 在 $(0,l)$ 内单调增加,证明其在 $(-l, 0)$ 内也单调增加。
  \item 函数 $f_1(x) = x^2(1-x^2)$、$f_2(x) = 3x^2 - x^3$、$f_3(x) = x(x-1)(x+1)$、$f_4(x) = (a^x + a^{-x})/2$、$f_5(x) = \sin x - \cos x + 1$ 中哪些是偶函数,哪些是奇函数,哪些既非奇函数也非偶函数?
  \item 函数 $f_1(x) = x \cos x$、$f_2(x) = \sin^2 x$、$f_3(x) = \cos(x-2)$、$f_4(x) = \cos (4x)$、$f_5(x) = 1 + \sin \pi x$ 中哪些是周期函数,对于周期函数,其最小正周期是多少?
  \item 函数 $f(x)$ 的定义域为 $[0,1]$,则函数 $f_1(x)=f(x^2)$、$f_2(x)=f(\sin x)$、$f_3(x)=f(x+a) + f(x-a)\ (a>0)$ 的定义域都是什么?
  \item 求函数 $f_1(x) = 1 + \ln(x+2)$、$f_2(x) = \sqrt[3]{x+1}$、$f_3(x) = \frac{2}{1+x}-1$、$f_4(x) = 1-\frac{1}{2^x + 1}$ 的反函数。
  \item 设函数 $f(x) = nx - m \ln x \ (n>0)$ 有两个零点,则 $\frac mn$ 的取值范围是什么? 
\end{problemset}

\chapter{数列极限}

\begin{introduction}
  \item 数列的定义
  \item 数列极限的描述性定义
  \item “$\varepsilon-N$ 语言”定义
  \item 如何证明一个数列的极限是 $a$
  \item 上、下(确)界
  \item 上下极限
\end{introduction}
\section{数列}
\begin{definition}[数列]


  与自然数一一对应的一系列实数就是\emph{数列}。记为 \emph{$\{a_n\}$}。在几何上,数列 $\{a_n\}$ 就是数轴上的一系列点。当然了,也可以把数列看成一种特殊的函数:$a_n = f(n), (\forall n \in \mathbb N^+) $。
\end{definition}

数列的每一个“$a_i$”叫做一\emph{项},其中的\emph{第 $n$ 项} $a_n$ 叫做数列的\emph{一般项}。数列中的项的个数称为这个数列的\emph{项数},一般情况下,数列应当至少有 $3$ 项。对于有无数个项的数列,我们称其为\emph{无穷项数列}。有时候会把无穷项数列 $a_1, a_2, a_3, \dots, a_n, \dots$ 写成 \emph{$\{a_n\}_1^{+\infty}$}。


\section{数列极限入门}
\subsection{描述性定义}
你可能会见过这样的题目:“当 $x$ 越来越大时,$f(x)$ 越来越接近多少?”。数列极限的描述性定义与之很像。从这样一道例题出发:

\begin{problem}
  随着 $n$ 的增大,$\sqrt{n^2+n}-n$ 越来越接近多少?
\end{problem}

显然对于这道题,我们可以考虑设一个数列,其通向公式为:$a_n = \sqrt{n^2+n}-n$,求这个数列的极限。由此出发,不难理解数列极限的描述性定义如下:

\begin{definition}[数列极限的描述性定义]
  已知一个数列 $a_n$,随着 $n$ 越来越大,$a_n$ 的值越来越接近的一个数即为\emph{数列的极限}。
\end{definition}

话是这么说,也不难理解。但是还是不严谨的。举个例子:

\begin{problem}
  对于一个数列 $\{a_n\}$,其通向公式 $a_n = f(n)$ 定义如下:当 $n$ 是质数时,$f(n)=1$;当 $n$ 不是质数时,$f(n)=\frac{1}{n}$。
\end{problem}

对于这个问题,随着 $n$ 越来越大,质数的密度越来越小,也就有越来越少的 $f(n)=1$。但是你不能说它就在 $n$ 足够大时没有任意一个 $f(n)=1$,那么你到底要怎么跟别人说明这个数列 $\{a_n\}$ 的极限是 $0$ 呢?显然这个数列的极限不存在。于是此时我们就只能引入 $\varepsilon-N$ 语言来帮忙了。

\newpage

\subsection{更加严格的定义}
\begin{definition}[数列极限的 $\varepsilon-N$ 语言定义]
  对于任意一个\emph{给定的}正数 $\varepsilon > 0$,无论其有多么小,总存在一个正数 $N$ 使得对于任意的 $n > N$,总能满足 $|a_n - p| < \varepsilon$。那么我们就称 $p$ 是数列 $\{a_n\}$ 的\emph{极限}。记为:

  \begin{equation*}
    \lim a_n = \lim_{n \rightarrow +\infty} a_n = \lim_{n \rightarrow \infty} a_n = p .
  \end{equation*}

  或 $a_n \rightarrow p(n \rightarrow + \infty) $。数列极限也可以这么描述:

  \begin{equation*}
    (\forall \varepsilon > 0)(\exists N > 0)(\forall n > N)(|a_n - p| < \varepsilon) .
  \end{equation*}
\end{definition}

回到刚才提到的,我们把数列看成数轴上的一系列点。那么应当有从 $N+1$ 项开始,以后的所有项对应的点都落在开区间 $(p-\varepsilon, p+\varepsilon)$ 内,这个数列的极限才为 $p$。

\subsection{数列极限的证明}
从一道例题入手:

\begin{problem}
  证明数列 $\{a_n\}$ 的极限是 $1$,$a_n = \frac{1+(-1)^{n-1}}{n}$。
\end{problem}

按照定义,我们只需证明,对于任意一个给定的正数 $\varepsilon > 0$,总能找到一个正数 $N$,使当 $n > N$ 时,$|a_n - 1| < \varepsilon$。

对于这类题目,我们先\emph{化简绝对值},比如该题化简得到 $\frac{1}n$。之后再\emph{化简绝对值的值小于 $\varepsilon$,得到一个关于 $n$ 的不等式},例如该题是化简 $\frac 1n < \varepsilon$,得到 $n > \frac{1}{\varepsilon}$。之后\emph{取一个满足条件的 $N$},例如该题是取 $N = \lfloor\frac{1}{\varepsilon}\rfloor$,当 $n > N$ 时,$|x_n - 1| = \frac{1}{n} < \varepsilon$。\emph{当然也可以取比算出的 $N$ 大一个正数的 $N'$},这样更保险。

下面是证明过程:
\begin{proof}
  由于 $|x_n - 1| = |1+\frac{(-1)^{n-1}}{n} - 1| = |\frac{(-1)^{n-1}}{n}| = \frac 1n$,对于任意的 $\varepsilon > 0$,为使 $|x_n - 1| < \varepsilon$,只需 $\frac{1}{n} < \varepsilon$ 即 $n > \varepsilon$。取 $N = \lfloor\frac{1}{\varepsilon}\rfloor$,当 $n > N$ 时,$|x_n - 1| = \frac 1n < \varepsilon$。所以该数列的极限为 $1$。
\end{proof}

\section{上下极限}
上下极限涉及到数列的上、下确界。为此我们先定义数列的上、下界:

\begin{definition}
  
\end{definition}

\newpage
\begin{problemset}
  \item 若 $\lim\limits_{n \rightarrow +\infty} {\frac{a_n-a}{a_n+a}}=0$, 证明:$\lim\limits_{n \rightarrow +\infty} a_n = a.$
  \item 若数列 $\lim\limits_{n \rightarrow +\infty} {\frac{a_n}n} = 0$, 证明:
    \begin{equation*}
      \lim\limits_{n \rightarrow +\infty} = \frac{\max\{a_1, a_2, \cdots, a_n\}}{n} = 0.
    \end{equation*}
  \item 若 $\lim a_n = a, \lim b_n = b$, 证明:
    \begin{equation*}
      \lim_{n \rightarrow +\infty} \frac{a_1b_n + a_2b_{n-1} + \cdots + a_nb_1}{n} = ab
    \end{equation*}
  \item 设 $\lim a_{2n} = a, \lim a_{2n-1} = b$, 证明:
    \begin{equation*}
      \lim_{n \rightarrow +\infty} \frac{a_1+a_2+\cdots+a_b}{n} = \frac{a+b}{2}.
    \end{equation*}
  \item 设 $\lim a_n = a$, 证明:
    \begin{equation*}
      \lim_{n \rightarrow +\infty} \frac{1}{2^n} \sum_{k=0}^n \binom{n}{k}a_k = a.
    \end{equation*}
  \item 设 $\lim\limits_{n \rightarrow +\infty} (x_n - x_{n-2}) = 0$, 证明:
    \begin{enumerate}
      \item
      \begin{equation*}
        \lim_{n \rightarrow +\infty} \frac{x_n}{n} = 0.
      \end{equation*}

      \item 
      \begin{equation*}
        \lim_{n \rightarrow +\infty} \frac{x_n - x_{n-1}}{n} = 0.
      \end{equation*}
    \end{enumerate}
    \item 若 $\lim\limits_{n \rightarrow +\infty} (x_n - x_{n-1})=d$, 证明:
      \begin{equation*}
        \lim_{n \rightarrow +\infty}\frac{x_n}{n} = d.
      \end{equation*}
\end{problemset}

\chapter{数列极限重难点}
\begin{introduction}
  \item 重要常数
  \item 常用不等式
  \item 常用命题、方法
  \item 夹逼定理
  \item 柯西收敛原理
\end{introduction}

这一章的内容会十分的多,因为数列极限的内容确实有难度且复杂,所以读者可以选择跳过这一章,直接进入下一章的内容。

\section{重要常数}
\begin{definition}[自然常数]
  \begin{equation*}
    e = \lim_{n \rightarrow +\infty} \left(1+\frac 1n\right)^n
  \end{equation*}
\end{definition}
\begin{definition}[欧拉常数]
  \begin{equation*}
    \gamma = \lim_{n \rightarrow +\infty} \left(1+\frac 12 + \frac 13 + \cdots + \frac 1n - \ln n\right)
  \end{equation*}
\end{definition}
\begin{definition}[Catalan 常数]
  \begin{equation*}
    G = \sum_{n=0}^{+\infty} \frac{(-1)^n}{(2n+1)^2}
  \end{equation*}
\end{definition}
\begin{definition}[Basel 问题]
  \begin{equation*}
    \sum_{n=1}^{+\infty} \frac{1}{n^2} = \frac{\pi ^2}{6}
  \end{equation*}
\end{definition}

\section{常用不等式}
\begin{theorem}[Jordan 不等式]
  \begin{equation*}
    \frac 2 \pi x \leq \sin x \leq x
  \end{equation*}

  其中 $0 \leq x \leq \frac \pi 2$。左侧等号成立当且仅当 $x = \frac \pi 2$, 右侧等号成立当且仅当 $x = 0$。
\end{theorem}
\begin{theorem}[柯西不等式]
  对于任意实数 $a_1, a_2, \cdots, a_n$ 和 $b_1, b_2, \cdots, b_n$, 都有:
  \begin{equation*}
    \left(\sum_{i=1}^n a_ib_i\right)^2 \leq \left(\sum_{i=1}^n a_i^2\right)\left(\sum_{i=1}^n b_i^2\right)
  \end{equation*}

  等号成立当且仅当 $b_i=0(i = 1,2,3\cdots, n)$ 或存在常数 $k$, 使得 $a_i = kb_i$。
\end{theorem}
\begin{theorem}[Wallis 不等式]
  \begin{equation*}
    \frac 1{\sqrt{\pi \left(n + \dfrac 12\right)}} < \frac{(2n-1)!!}{(2n)!!} < \frac{1}{\sqrt{\pi n}}
  \end{equation*}
  
  其中 $n \in \mathbb{N}^+$,$(2n)!! = 2 \times 4 \times 6 \times \dots \times (2n)$,$(2n-1)!! = 1 \times 3 \times 5 \times \dots \times (2n-1)$。
  
  有时 Wallis 公式也代表下面这个极限:

  \begin{equation*}
    \lim_{n \rightarrow +\infty} \left(\frac{(2n)!!}{(2n-1)!!}\right)^2 \frac{1}{2n+1} = \frac{\pi}2
  \end{equation*}
\end{theorem}
\begin{theorem}[平均值不等式]
  设 $a_1, a_2, \dots, a_n$ 是 $n$ 个非负实数,则
  \begin{equation*}
    \frac{a_1+a_2+\cdots+a_n}n \geq \sqrt[n]{a_1a_2 \cdots a_n}
  \end{equation*}
  等号成立当且仅当 $a_1 = a_2 = \cdots = a_n$。

  拓展版:

  $$\frac{n}{\sum _{i=1}^n \frac{1}{a_i}} \leq \sqrt[n]{\prod_{i=1}^n a_i} \leq \frac{a_1 + a_2 + \cdots + a_n}{n} \leq \sqrt{\frac{a_1^2 + a_2^2 + a_3^3 + \cdots + x_n^2}{n}}$$

  同样的,等号成立的充要条件是 $a_1 = a_2 = \cdots = a_n$。

  对于二元的,有时还会涉及到对数平均数, 两个数的对数平均数 $L(a,b)$ 定义为
  \begin{equation*}
    L(a,b) = \frac{a-b}{\ln a-\ln b} \qquad (a \neq b)
  \end{equation*}

  \begin{equation*}
    L(a,b) = a \qquad (a = b)
  \end{equation*}

  则有:

  \begin{equation*}
    \sqrt{ab} \leq L(a,b) \leq \frac{a+b}2
  \end{equation*}
\end{theorem}
\begin{theorem}[三角不等式]
  对于任意实数 $a$ 和 $b$,都有
  
  $$||a| - |b|| \leq |a+b| \leq |a| + |b|$$

  左边等号成立当且仅当 $ab \leq 0 $,右边等号成立当且仅当 $ab \geq 0$。
  
  同时也有:

  $$||a|-|b|| \leq |a-b| \leq |a| + |b|$$

  左边等号成立当且仅当 $ab \geq 0$, 右边等号成立当且仅当 $ab \leq 0$。
\end{theorem}
\newpage
\begin{theorem}[排序不等式]
  设有两个数列:$a_1, a_2, \dots, a_n$ 和 $b_1, b_2, \dots, b_n$ 满足 $a_1 \leq a_2 \leq \cdots \leq a_n$, $b_1 \leq b_2 \leq \cdots \leq b_n$,数列 $\{c_i\}_1^n$ 是 $\{b_i\}_1^n$ 的乱序排列,定义如下:
  
  \begin{itemize}
    \item 顺序和为 $\sum\limits_{i=1}^{n} {a_ib_i}$, 不妨记为 $S_1$;
    \item 乱序和为 $\sum\limits_{i=1}^{n} {a_ic_i}$, 不妨记为 $S$;
    \item 逆序和为 $\sum\limits_{i=1}^{n} {a_ib_{n-i+1}} = a_1b_n + a_2b_{n-1} + \cdots + a_{n-1}b_2 + a_nb_1$, 不妨记为 $S_2$。
  \end{itemize}

  则有:$S_1 \geq S \geq S_2$(顺序和 $\geq$ 乱序和 $\geq$ 逆序和)。

  等号成立当且仅当 $a_1 = a_2 = \cdots = a_n$ 或 $b_1=b_2=\cdots=b_n$。

  此不等式应用较少,但是可以用于竞赛和对于数列的前几项找规律,值得一学。
\end{theorem}
还有以下不等式较为常用:
\begin{proposition}
  \begin{equation*}
    \frac{x}{1+x} \leq \ln(1+x) \leq x
  \end{equation*}

  其中 $x > -1$, 两边等号均成立当且仅当 $x=0$。
\end{proposition}
\begin{proposition}
  \begin{equation*}
    \sin x \leq x \leq \tan x
  \end{equation*}

  其中 $0 \leq x < \frac{\pi}2$, 两边等号均成立当且仅当 $x=0$。
\end{proposition}
\begin{proposition}
  $$\left(\frac{n+1}{e}\right)^n < n! < e\left(\frac{n+1}e\right)^{n+1}$$
\end{proposition}
\begin{proposition}
  $$\frac{1}{n+1} < \ln\left(1+\frac1n\right) < \frac 1n$$

  其中 $n \in \mathbb N^+$。
\end{proposition}
\begin{proposition}
  \begin{equation*}
    e^x \geq x+1
  \end{equation*}

  其中 $x \in \mathbb R$, 等号成立当且仅当 $x=0$。
\end{proposition}

\section{审敛与计算}

夹逼定理十分常用, 希望读者能熟练掌握。柯西收敛原理也有一定的应用范围, 可以适当练习。本章章后练习会偏多, 读者可以选择性地练习, 如遇看不懂解析的欢迎加入 QQ 群或与笔者讨论。

\subsection{夹逼定理}
\begin{theorem}[夹逼定理]
  若有三个数列 $\{a_n\}, \{b_n\}, \{c_n\}$ 从某项开始有 $a_n \leq b_n \leq c_n, \forall n > N_0$。
  
  那么若有

  $$\lim_{n \rightarrow +\infty} a_n = \lim_{n \rightarrow +\infty} c_n = t$$

  则

  $$\lim_{n \rightarrow +\infty} b_n = t.$$
\end{theorem}
\subsection{Cauchy 收敛原理}
\begin{definition}[基本数列]
  如果数列 $\{a_n\}$ 具有性质如下:

  $$(\forall \varepsilon>0)(\exists N \in \mathbb N^+)(\forall n, m > N)(|a_n - a_m| < \varepsilon)$$

  则称数列 $\{a_n\}$ 是一个基本数列。

\end{definition}
\begin{theorem}[Cauchy 收敛原理]
  数列 $\{a_n\}$ 收敛当且仅当它是基本数列。
\end{theorem}

\section{常用命题、方法}

\begin{proposition}[Abel 变换]
  设 $\{a_n\}, \{b_n\}$ 是两个数列, 记 $S_k = b_1 + b_2 + \cdots + b_k (k = 1,2,\cdots)$, 则有:

  \begin{equation*}
    \sum_{k=1}^p a_kb_k = a_pS_p - \sum_{k=1}^{p-1} (a_{k+1} - a_k) S_k.
  \end{equation*}
\end{proposition}
\begin{theorem}[“拉链”定理]
  设 $\lim x_{2n} = \lim x_{2n+1} = a$, 则 $\lim x_n = a$。
\end{theorem}
\begin{theorem}[单调有界定理]
  单调递增且有上界的数列必定收敛;单调递减且有下界的数列必定收敛。
\end{theorem}
\begin{proposition}[Cauchy 命题]
  说是命题, 把它当作定理来用也未尝不可。
  
  Cauchy 命题的描述是这样的:若有数列 $\{a_n\}, \lim a_n = a$, 则有
  
  $$\lim\limits_{n \rightarrow +\infty} \frac{a_1 + a_2 + \cdots + a_n}{n} = a.$$
\end{proposition}
\begin{proposition}[Wallis 公式]
  这里只再补充两个:

  \begin{equation*}
    \frac{(2n)!!}{(2n-1)!!} \sim \sqrt{\pi n} (n \rightarrow +\infty)
  \end{equation*}

  \begin{equation*}
    \frac{(n!)^2 2^{2n}}{(2n)!} \sim \sqrt{\pi n} (n \rightarrow +\infty).
  \end{equation*}
\end{proposition}
\begin{proposition}[Stirling 公式]
  由于这个公式形式多变, 较为重要, 所以这里再完整地描述一遍。

  $$n! = \sqrt{2\pi n}\left(\frac ne\right)^n e^{\frac{\theta_n}{12 n}}, 0 < \theta_n < 1$$

  或以下几个形式:

  \begin{equation*}
    \ln n! = \ln \sqrt{2\pi} + \left(n + \frac 12\right)\ln n - n + \frac{\theta_n}{12n}, 0 < \theta_n < 1
  \end{equation*}

  \begin{equation*}
    n! \sim \sqrt{2 \pi n} \left(\frac ne\right)^n (n \rightarrow +\infty)
  \end{equation*}

  \begin{equation*}
    n! = \sqrt{2 \pi n} \left(\frac ne\right)^n \left(1 + \frac{1}{12n} + \frac{1}{288 n^2} - \frac{129}{51840n^3} - \frac{571}{2488320n^4} + \cdots\right)
  \end{equation*}

  \begin{equation*}
    \ln n! = \ln \sqrt{2\pi} + \left(n + \frac 12\right) \ln n - n + \frac{B_2}{1 \cdot 2n} + \frac{B_4}{3 \cdot 4n^3} + \cdots + \frac{B_{2m}}{(2m-1)(2m)n^{2m-1}} + \frac{B_{2m+2}}{(2m+1)(2m+2)n^{2m+1}}\theta_n
  \end{equation*}

  值得注意的是,在最后的一个形式中,$0 < \theta_n < 1$, $B_{2n}$ 是 Bernoulli 数。

  涉及 $n!$ 时应当优先考虑 Stirling 公式。

\end{proposition}
\begin{theorem}[Toeplitz 定理]
  设 $n, k \in \mathbb N^+$, $t_{nk} \geq 0$。且有:

  $$\sum_{k=1}^n t_{nk} = 1, \lim_{n \rightarrow +\infty} t_{nk} = 0.$$

  若 $\lim a_n = a$, 则有:

  $$\lim_{n \rightarrow +\infty} \sum_{k=1}^n t_{nk} a_k = a.$$

  值得注意的是:

  \begin{enumerate}
    \item 若将条件 $\sum\limits_{k=1}^n t_{nk} = 1$ 修改为 $\lim \limits_{n \rightarrow +\infty} \sum\limits_{k=1}^n t_{nk}=1$ 且 $a=0$, 结论仍成立;
    \item 若不要求 $t_{nk}$ 非负,则将条件 $\sum\limits_{k=1}^n t_{nk} = 1$ 改为存在 $M>0$, 使得对每个正整数 $n$ 成立 $|t_{n1}|+|t_{n2}|+\cdots+|t_{nk}| \leq < M$ 且 $a=0$, 结论仍成立。
  \end{enumerate}

\end{theorem}

Toeplitz 定理的证明并不难,但结论十分强大!令 $t_{nk} = \frac 1n$ 就可以推出柯西命题,也可以快速证明 Stolz 定理等。

这些方法都可以应用于对于原数列的放缩等、最后运用夹逼定理(或“两面包夹芝士”定理)求得极限。
\newpage
\begin{problemset}
  \item 咕咕咕。
\end{problemset}
\chapter{习题解答}
\section{预备知识相关习题解答}
\subsection{第二章习题解答}
\begin{enumerate}
\item 显然不同,第二个函数的 $x$ 取不到负数,而第一个可以。
\item 单调增加。 $f(x) = -1 + \frac1{1-x}$,设 $x_1 < x_2 < 1$,因为 $f(x_2) - f(x_1) = \frac{x_2-x_1}{(1-x_1)(1-x_2)} > 0$,所以 $f(x_2) > f(x_1)$,即 $f(x)$ 在 $(-\infty, 1)$ 内单调增加。
\item 单调增加。设 $0 < x_1 < x_2$,因为 $f(x_2) < f(x_1) = x_2 + \ln x_2 - x_1 - \ln x_1 = x_2 - x_1 + \ln \frac{x_2}{x_1} > 0$,所以 $f(x_2) > f(x_1)$,即 $f(x)$ 在 $(0, +\infty)$ 内单调增加。
\item 设 $-l < x_1 < x_2 < 0$,则 $0 < -x_2 < -x_1 < l$,由于 $f(x)$ 是奇函数,所以 $f(x_2) - f(x_1) = -f(-x_2) + f(-x_1)$。因为 $f(x)$ 在 $(0, l)$ 上单调增加,所以 $f(-x_1) - f(-x_2) > 0$,即 $f(x_2) > f(x_1)$,所以 $f(x)$ 在 $(-l, 0)$ 内也单调增加。
\item 显然 $f_1(x)$ 是偶函数,$f_2(x)$ 既非奇函数也非偶函数、$f_3(x)$ 是奇函数、$f_4(x)$ 是偶函数、$f_5(x)$ 既非奇函数也非偶函数。
\item 显然 $f_1(x)$ 不是周期函数,$f_2(x)$ 是周期函数、其最小正周期为 $\pi$,$f_3(x)$ 是周期函数、其最小正周期为 $2\pi$,$f_4(x)$ 是周期函数、其最小正周期为 $\frac{\pi}2$,$f_5(x)$ 是周期函数、其最小正周期为 $2$。
\item $[-1, 1]$,$[2n \pi, (2n+1) \pi]\ (n \in \mathbb{Z})$,当 $0 < a \leq 0.5$ 时为 $[a, 1-a]$、当 $a > 0.5$ 时定义域为 $\varnothing$。
\item $y = e^{x-1}-2$、$y=x^3-1$、$y=\frac{1-x}{1+x}$、$y = \log_2 \frac{x}{1-x}$。
\item $(e, +\infty)$。令 $nx - m \ln x = 0$,解得 $\frac m n = \frac{\ln x}x$,易知 $g(x) = \frac{\ln x}{x}$ 在 $(0, e)$ 上单调递增,在 $(e, +\infty)$ 上单调递减。注意到 $g(0^+) = -\infty$,$f(+\infty) = 0$,$g(e) = \frac 1e$。要想其有两根,则 $0 < \frac m n < \frac 1e$,即 $\frac m n > e$,即 $(e, +\infty)$。
\end{enumerate}


\section{极限相关习题解答}
\subsection{第三章习题解答}
本章习题全部可以使用 $\varepsilon-N$ 语言证明,故省略。
\subsection{第四章习题解答}


\appendix

\chapter{版本更新历史}

根据同学以及朋友的反馈(以及笔者的自我批判),经过不断地修改,本书已有了将近 $40$ 次的更新、修改、Debug。十分感谢大家的资瓷!

\datechange{2022/3/21}{版本 1.0 正式发布}
\begin{change}
  \item 初版发布
\end{change}
番外:这天是我前桌的生日。结果我忘了给她准备生日礼物了。

\datechange{2022/3/22}{版本 1.1 正式发布}
\begin{change}
  \item 更换封面、使用 Elegant 模板、使用 \LaTeX 重新排版。
  \item 完善第二章函数相关内容(添加单调性、奇偶性)。
\end{change}
番外:这天是我班主任生日,结果我们班没一个同学知道。

\datechange{2022/3/23}{版本 1.1.1a 发布}
\begin{change}
  \item 完善函数相关内容(添加有界性、周期性)。
\end{change}

\datechange{2022/3/25}{版本 1.1.1b 发布}
\begin{change}
  \item 添加第二章的习题。
\end{change}

\datechange{2022/3/26}{版本 1.1.2 发布}
\begin{change}
  \item 添加第二章习题第 9 题。
  \item 添加 2.1:几何与逻辑符号。
\end{change}

\datechange{2022/3/28}{版本 1.1.3a 发布}
\begin{change}
  \item 添加第二章习题前 $5$ 题解答。
\end{change}

\datechange{2022/3/29}{版本 1.1.3b 发布}
\begin{change}
  \item 添加第二章后面全部习题的解答。
  \item 完善格式。
\end{change}

\datechange{2022/3/31}{版本 1.2 发布}
\begin{change}
  \item 添加第一章。
  \item 开始写第三章(数列极限)。
\end{change}

\datechange{2022/4/2}{版本 1.3.1 发布}
\begin{change}
  \item 添加版本历史更新。
  \item 编写 3.1:数列。
\end{change}

\datechange{2022/4/3}{版本 1.3.2 发布}
\begin{change}
  \item 完善第一章。
\end{change}

\datechange{2022/4/5}{版本 1.4.1 发布}
\begin{change}
  \item 编写数列极限的定义相关内容。
  \item 编写数列极限的证明相关内容。
  \item 完善格式。
\end{change}

\datechange{2022/4/6}{版本 1.5 发布}
\begin{change}
  \item 编写“数列极限重难点”一章。\LaTeX 源码大概增加了 $7$k。
  \item 咕咕咕。
\end{change}

\datechange{2022/4/7}{版本 1.5.1 发布}
\begin{change}
  \item 添加第三章习题。
  \item 第四章添加排序不等式、重新排版。
\end{change}

\datechange{2022/4/9}{版本 1.5.2 发布}
\begin{change}
  \item 添加(数列)上下极限。
\end{change}
\end{document}
